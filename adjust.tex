%%%%%%%%%%%%%%%%%%%%%%%% GNU General Public License 3.0 %%%%%%%%%%%%%%%%%%%%%%%%
%%                                                                            %%
%% Copyright (C) 2023 Kevin Matthes                                           %%
%%                                                                            %%
%% This program is free software: you can redistribute it and/or modify       %%
%% it under the terms of the GNU General Public License as published by       %%
%% the Free Software Foundation, either version 3 of the License, or          %%
%% (at your option) any later version.                                        %%
%%                                                                            %%
%% This program is distributed in the hope that it will be useful,            %%
%% but WITHOUT ANY WARRANTY; without even the implied warranty of             %%
%% MERCHANTABILITY or FITNESS FOR A PARTICULAR PURPOSE.  See the              %%
%% GNU General Public License for more details.                               %%
%%                                                                            %%
%% You should have received a copy of the GNU General Public License          %%
%% along with this program.  If not, see <https://www.gnu.org/licenses/>.     %%
%%                                                                            %%
%%%%%%%%%%%%%%%%%%%%%%%%%%%%%%%%%%%%%%%%%%%%%%%%%%%%%%%%%%%%%%%%%%%%%%%%%%%%%%%%

%%%%%%%%%%%%%%%%%%%%%%%%%%%%%%%%%%%%%%%%%%%%%%%%%%%%%%%%%%%%%%%%%%%%%%%%%%%%%%%%
%%
%%  AUTHOR      Kevin Matthes
%%  BRIEF       This project's manual.
%%  COPYRIGHT   GPL-3.0
%%  DATE        2023
%%  FILE        adjust.tex
%%  NOTE        See `LICENSE' for full license.
%%              See `README.md' for project details.
%%
%%%%%%%%%%%%%%%%%%%%%%%%%%%%%%%%%%%%%%%%%%%%%%%%%%%%%%%%%%%%%%%%%%%%%%%%%%%%%%%%

\documentclass[11pt, a4paper, british]{scrartcl}

\usepackage{asycolors}
\usepackage[british]{babel}
\usepackage[T1]{fontenc}
\usepackage
[ bottom  = 3 cm
, inner   = 3 cm
, outer   = 3 cm
, top     = 3 cm
, twoside
]{geometry}
\usepackage[utf8]{inputenc}
\usepackage{lmodern}
\usepackage{url}

\usepackage{enumitem}
\setlist{nosep}

\usepackage[english]{varioref}
\usepackage[bookmarks, bookmarksnumbered, bookmarksopen, hidelinks]{hyperref}
\usepackage[english]{cleveref}

\usepackage{booktabs}
\usepackage{listings}
\lstset{
    backgroundcolor = \color{lightgray},
    basicstyle      = \ttfamily,
    keywordstyle    = \color{heavygreen},
    numbers         = left,
    numberstyle     = \tiny
}

\Crefname{lstlisting}{Listing}{Listings}
\crefname{lstlisting}{listing}{listings}

\DeclareRobustCommand{\adjust}{\textsf{adjust}}

\author{Kevin Matthes}
\date{\today}
\subtitle{Yet Another Text Editor for the Terminal, Written in V}
\title{The \texttt{adjust} Text Editor}

%%%%%%%%%%%%%%%%%%%%%%%%%%%%%%%%%%%%%%%%%%%%%%%%%%%%%%%%%%%%%%%%%%%%%%%%%%%%%%%%

\begin{document}
\maketitle
\tableofcontents
\listoftables

%%%%%%%%%%%%%%%%%%%%%%%%%%%%%%%%%%%%%%%%%%%%%%%%%%%%%%%%%%%%%%%%%%%%%%%%%%%%%%%%

\newpage
\section{Installation and Updating}
\label{sec:installation-and-updating}
\subsection{Installation}
\label{sec:installation}
At the moment, the only way to install \adjust\ is from source.  In order to
create a local copy of the public GitHub repository, it is recommended to
execute the command shown in \Vref{lst:download} to clone the entire repository
to \texttt{\textasciitilde/.vmodules/adjust/}.

\begin{lstlisting}[caption = Downloading \adjust, label = lst:download]
v install --git https://github.com/kevinmatthes/adjust
\end{lstlisting}

The second step is the compilation of the executable.  Users should compile the
software using the instruction given in \Vref{lst:compilation} in order to
achieve an enhanced performance (flag \texttt{-prod}).  The flag \texttt{-W}
will make sure that the source code does not contain any errors.

\begin{lstlisting}[caption = Compiling \adjust, label = lst:compilation]
v -W -prod ~/.vmodules/adjust/
\end{lstlisting}

For convenience, the resulting binary may be linked to \texttt{\$PATH}, at
option.  Unix users might therefore consider the command stated in \Vref
{lst:softlink} useful.

\begin{lstlisting}[caption = Softlinking \adjust, label = lst:softlink]
ln -s ~/.vmodules/adjust/adjust ~/.local/bin/adjust
\end{lstlisting}

\subsection{Self-Updating}
\label{sec:self-updating}
\adjust\ has a built-in self-updating mechanism which is defined as a wrapper
around \Vref{lst:download} and \Vref{lst:compilation}.  It is invoked as
illustrated by \Vref{lst:update}.

\begin{lstlisting}[caption = Updating \adjust, label = lst:update]
adjust --nightly
\end{lstlisting}

%%%%%%%%%%%%%%%%%%%%%%%%%%%%%%%%%%%%%%%%%%%%%%%%%%%%%%%%%%%%%%%%%%%%%%%%%%%%%%%%

\newpage
\section{Editor Modes}
\label{sec:editor-modes}
\subsection{Command Mode}
\label{sec:command-mode}
The \texttt{command} mode of \adjust\ is designed to instruct the editor in a
terminal-like manner.

If the current editor mode is \texttt{view}, the \texttt{command} mode can be
entered by typing \texttt{:}.  Any succeeding text input will be added to the
command buffer whilst the backspace key will remove the last character from it.
The current content of the command buffer is shown in the second line of the
status bar.  When pressing the enter key, the specified command will be executed
which usually involves resetting the command buffer.  Unknown commands will not
have any effect.

The supported commands are explained in the following by their category.

\subsubsection{Changing to \texttt{insert} Mode}
\label{sec:changing-to-insert-mode}
To enter the \texttt{insert} mode from the \texttt{command} mode, users might
consider the following commands useful in addition to entering the target mode
by changing to \texttt{view} mode intermediately.

\begin{itemize}
\item \texttt{:edit}
\item \texttt{:insert}
\end{itemize}

\subsubsection{Changing to \texttt{view} Mode}
\label{sec:changing-to-view-mode}
In order to leave the \texttt{command} mode and to go back to the \texttt{view}
mode, not only pressing the escape key will have the intended effect but also
the following commands.

\begin{itemize}
\item \texttt{:cancel}
\item \texttt{:view}
\end{itemize}

\subsubsection{Leaving the Editor}
\label{sec:leaving-the-editor}
These commands can be used to leave the current session.  Exiting a session will
not only terminate the process but also save and reformat the current file, if
available.

The supported commands are as follows.

\begin{itemize}
\item \texttt{:exit}
\item \texttt{:leave}
\item \texttt{:quit}
\end{itemize}

\subsubsection{Leaving the Editor without Saving}
\label{sec:leaving-the-editor-without-saving}
Sometimes, the changes made to a file shall not be kept.  Therefore, these
commands might come in handy.  Neither will the changes be saved nor the file be
reformatted, if available at all.

The supported commands are:

\begin{itemize}
\item \texttt{:exit unchanged}
\item \texttt{:leave unchanged}
\item \texttt{:quit unchanged}
\end{itemize}

At the moment, this is the sole possibility to leave \adjust\ without persisting
changes to a file.

\subsubsection{Reformatting the Current File}
\label{sec:reformatting-the-current-file}
For many programming languages, formatting tools are available in order to meet
the configured format conventions.  As it is assumed that editing code of a
particular language requires their toolchains being installed, \adjust\ will
\emph{always} reformat source files of the supported languages when they are
saved.  To trigger the reformatting manually, the following commands are
available.

\begin{itemize}
\item \texttt{:format}
\item \texttt{:reformat}
\end{itemize}

Please see also:

\begin{itemize}
\item \Vref{sec:f5-reformat}
\item \Vref{tab:formatters}
\end{itemize}

\subsubsection{Saving the Current File}
\label{sec:saving-the-current-file}
It is recommended to save the changes made to a file as often as possible in
order to prevent loss of progress.  \adjust\ supports its users in doing so by
automatically saving whenever possible and sensible.  Furthermore, the current
file can be saved by the following commands manually.

\begin{itemize}
\item \texttt{:save}
\item \texttt{:write}
\end{itemize}

Please see also:

\begin{itemize}
\item \Vref{sec:f2-save}
\end{itemize}

%%%%%%%%%%%%%%%%%%%%%%%%%%%%%%%%%%%%%%%%%%%%%%%%%%%%%%%%%%%%%%%%%%%%%%%%%%%%%%%%

\newpage
\section{Key Bindings}
\label{sec:key-bindings}
\subsection{\texttt{F2}:  Save}
\label{sec:f2-save}
To save the recent changes, users just need to press \texttt{F2}.  The recent
changes will then be written to the connected file path which will create a new
file in case that it did not exist, yet.

This key binding works in \emph{every} editor mode.

\subsection{\texttt{F5}:  Reformat}
\label{sec:f5-reformat}
To reformat the currently opened file, users just need to press \texttt{F5}.
The recent changes will then be written to the connected file path, the selected
formatter will be run over the connected file path, and the file contents will
be reloaded.  Thereby, the view will be reset.

At the moment, the formatters are hard-coded into the binary and are usually run
as separate processes in the background.  \adjust\ will always wait for the
formatters to return, so this might take some time for large files and / or
projects.  The formatters will be run silently, i.e., failures and / or error
messages, if any, are \emph{not} going to be reported back to the user.

\Vref{tab:formatters} gives a complete list of the supported formatters, sorted
by the respective computer language in the left-hand column.  The right-hand
column shows the executed command where \texttt{$f$} denotes the currently opened
file.

\begin{table}
\centering
\caption{Supported formatters}
\label{tab:formatters}
\begin{tabular}{cl}
\toprule
Computer language   & Executed command          \\
\midrule
\LaTeX              & \texttt{latexmk}          \\
Nim                 & \texttt{nimpretty $f$}    \\
Rust                & \texttt{rustfmt $f$}      \\
V                   & \texttt{v fmt -w $f$}     \\
\bottomrule
\end{tabular}
\end{table}

This key binding works in \emph{every} editor mode.

%%%%%%%%%%%%%%%%%%%%%%%%%%%%%%%%%%%%%%%%%%%%%%%%%%%%%%%%%%%%%%%%%%%%%%%%%%%%%%%%

\end{document}

%%%%%%%%%%%%%%%%%%%%%%%%%%%%%%%%%%%%%%%%%%%%%%%%%%%%%%%%%%%%%%%%%%%%%%%%%%%%%%%%
