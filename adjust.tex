%%%%%%%%%%%%%%%%%%%%%%%% GNU General Public License 3.0 %%%%%%%%%%%%%%%%%%%%%%%%
%%                                                                            %%
%% Copyright (C) 2023 Kevin Matthes                                           %%
%%                                                                            %%
%% This program is free software: you can redistribute it and/or modify       %%
%% it under the terms of the GNU General Public License as published by       %%
%% the Free Software Foundation, either version 3 of the License, or          %%
%% (at your option) any later version.                                        %%
%%                                                                            %%
%% This program is distributed in the hope that it will be useful,            %%
%% but WITHOUT ANY WARRANTY; without even the implied warranty of             %%
%% MERCHANTABILITY or FITNESS FOR A PARTICULAR PURPOSE.  See the              %%
%% GNU General Public License for more details.                               %%
%%                                                                            %%
%% You should have received a copy of the GNU General Public License          %%
%% along with this program.  If not, see <https://www.gnu.org/licenses/>.     %%
%%                                                                            %%
%%%%%%%%%%%%%%%%%%%%%%%%%%%%%%%%%%%%%%%%%%%%%%%%%%%%%%%%%%%%%%%%%%%%%%%%%%%%%%%%

%%%%%%%%%%%%%%%%%%%%%%%%%%%%%%%%%%%%%%%%%%%%%%%%%%%%%%%%%%%%%%%%%%%%%%%%%%%%%%%%
%%
%%  AUTHOR      Kevin Matthes
%%  BRIEF       This project's manual.
%%  COPYRIGHT   GPL-3.0
%%  DATE        2023
%%  FILE        adjust.tex
%%  NOTE        See `LICENSE' for full license.
%%              See `README.md' for project details.
%%
%%%%%%%%%%%%%%%%%%%%%%%%%%%%%%%%%%%%%%%%%%%%%%%%%%%%%%%%%%%%%%%%%%%%%%%%%%%%%%%%

\documentclass[11pt, a4paper, british]{scrartcl}

\usepackage{asycolors}
\usepackage[british]{babel}
\usepackage[T1]{fontenc}
\usepackage
[ bottom  = 3 cm
, inner   = 3 cm
, outer   = 3 cm
, top     = 3 cm
, twoside
]{geometry}
\usepackage[utf8]{inputenc}
\usepackage{lmodern}
\usepackage{url}

\usepackage{enumitem}
\setlist{nosep}

\usepackage[english]{varioref}
\usepackage[bookmarks, bookmarksnumbered, bookmarksopen, hidelinks]{hyperref}
\usepackage[english]{cleveref}

\usepackage{listings}
\lstset{
    backgroundcolor = \color{lightgray},
    basicstyle      = \ttfamily,
    keywordstyle    = \color{heavygreen},
    numbers         = left,
    numberstyle     = \tiny
}

\DeclareRobustCommand{\adjust}{\textsf{adjust}}

\author{Kevin Matthes}
\date{\today}
\subtitle{Yet Another Text Editor for the Terminal, Written in V}
\title{The adjust Text Editor}

\begin{document}
\maketitle
\tableofcontents

\section{Installing and Updating the Editor}
\label{sec:installing-and-updating}
It is recommended to install \adjust\ from source using the V CLI with the
following command.

\begin{lstlisting}[caption = Installing \adjust\ from source, language = bash]
v install --git https://github.com/kevinmatthes/adjust
\end{lstlisting}

Once installed, \adjust\ can be updated using either the above command or
its built-in self-update mechanism which is technically just a wrapper around
the above instruction to the V CLI.

\begin{lstlisting}[caption = Updating \adjust, language = bash]
adjust --nightly
\end{lstlisting}

\section{Working with the Editor}
\label{sec:working}
\subsection{Starting the Editor}
\label{sec:working:starting}
As soon as the editor is installed, users can begin to edit their files with
\adjust.  To do so, the editor must be called with the paths to the files to
edit during the session.  Files which do not exist, yet, are going to be created
on save.

\begin{lstlisting}[caption = Starting \adjust, language = bash]
adjust a.v b.c c.nim d.v
\end{lstlisting}

\subsection{Leaving the Editor}
\label{sec:working:leaving}
To leave the current session, users will need to change to the command mode and
execute one of the following commands.

\begin{itemize}
\item \texttt{:exit}
\item \texttt{:exit unchanged}
\item \texttt{:leave}
\item \texttt{:leave unchanged}
\item \texttt{:quit}
\item \texttt{:quit unchanged}
\end{itemize}

All commands are synonymous regarding their final effect:  quitting the current
editing session.  Those commands ending in \texttt{unchanged} will cause the
\adjust\ to discard all changes while those \emph{not} ending in \texttt
{unchanged} will automatically save all changes before leaving.
\end{document}

%%%%%%%%%%%%%%%%%%%%%%%%%%%%%%%%%%%%%%%%%%%%%%%%%%%%%%%%%%%%%%%%%%%%%%%%%%%%%%%%
